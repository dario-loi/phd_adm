% This is a LaTeX template for an ICSC submission, based on a homework template by Lucas R. Ximenes (Jimeens)
% Refactored for Dario Loi, M.Sc Computer Science, Sapienza University of Rome

%%%%%%%%%%%%%%%%%%%%%%%%%%%%%%%%%%%%%%%%%%%%%%%%%%%%%%%%%%%%%%%%%

\documentclass{solutionclass} % Using the original class as requested
\usepackage{listings} % Add the listings package for code highlighting
\usepackage{xcolor} % Required for defining custom colors

% Define a custom Python style for the code listing
\lstdefinestyle{pythonstyle}{
    language=Python,
    backgroundcolor=\color{gray!10},
    commentstyle=\color{blue!30!black},
    keywordstyle=\color{red!75!black},
    numberstyle=\tiny\color{gray},
    stringstyle=\color{green!40!black},
    basicstyle=\ttfamily\footnotesize,
    breakatwhitespace=false,
    breaklines=true,
    captionpos=b,
    keepspaces=true,
    numbers=left,
    numbersep=5pt,
    showspaces=false,
    showstringspaces=false,
    showtabs=false,
    tabsize=2
}


\pagestyle{plain}

\begin{document}

\pretitle
{ICSC 2025 Submission}                          % ⟸ Main Title
{Solutions to the Qualification Round}          % ⟸ Subtitle
{Dario Loi} % ⟸ Author and Affiliation

\makeatletter
    \startcontents[sections]
    \phantomsection
    \chapter{Solutions}
\makeatother

    This document presents the solutions for the qualification round of the International Computer Science Competition (ICSC) 2025. Each section corresponds to a problem from the problem sheet. For programming problems, the source code is provided with a description and in-line comments, as per the submission guidelines. [1]

    \divider

    \section{Problem 1: Conceptual}
    Here you can write the solution to the first conceptual problem. Explain your reasoning and the steps you took to arrive at the answer.

    \begin{solution}[Solution to Problem 1]
        This is where the detailed solution would go. For example, if the problem was about Big O notation, you could explain the analysis of the algorithm's complexity here.
    \end{solution}

    \section{Problem 2: Programming}
    Here you would describe your approach to the programming problem.

    \begin{solution}[Algorithm Description]
        First, I identified the core requirements of the problem. My approach is to use a recursive backtracking algorithm to explore the solution space. The base case for the recursion is when a valid solution is found. I've chosen Python for its readability and extensive libraries.
    \end{solution}

    \begin{solution}[Source Code: Python]
\begin{lstlisting}[style=pythonstyle]
# Your Python code with in-line comments would go here
def solve_problem_2(data):
    """
    This function implements the solution for Problem 2.
    It takes 'data' as input and returns the result.
    """
    # Step 1: Initialize variables
    result = 0

    # Step 2: Main logic of the algorithm
    for item in data:
        # Process each item
        result += item

    # Step 3: Return the final result
    return result

# Example usage:
# input_data = [1, 2, 3, 4, 5]
# print(solve_problem_2(input_data))
\end{lstlisting}
    \end{solution}

    \section{Problem 3: Advanced Topic}
    This section could be for a more advanced problem, perhaps involving topics like machine learning or cryptography. [2]

    \begin{solution}[Solution to Problem 3]
        The solution to this problem involves applying the RSA encryption algorithm. The following equations demonstrate the key generation and encryption process.
        \begin{equation}\label{eq: 1}
            c \equiv m^e \pmod{n}
        \end{equation}
        Where 'c' is the ciphertext, 'm' is the message, and '(e, n)' is the public key.
    \end{solution}

    \divider

    \section{Concluding Remarks}
    This submission presents my solutions to the ICSC 2025 Qualification Round problems. The solutions have been explained, and the code has been commented to ensure clarity and reproducibility.

%%%%%%%%%%%%%%%%%%%%%%%%%%%%%%%%%%%%%%%%%%%%%%%%%%%%%%%%%%%%
    % The following commands add a footer with the original template author's credit.
    % You may choose to keep or remove it.
    \thispagestyle{fancyplain}
    \fancyhead{}
    \renewcommand{\headrulewidth}{0pt}
    \rfoot{Template by L. R. Ximenes (Jimeens)}
%%%%%%%%%%%%%%%%%%%%%%%%%%%%%%%%%%%%%%%%%%%%%%%%%%%%%%%%%%%%
\end{document}